\PassOptionsToPackage{unicode=true}{hyperref} % options for packages loaded elsewhere
\PassOptionsToPackage{hyphens}{url}
%
\documentclass[]{article}
\usepackage{lmodern}
\usepackage{amssymb,amsmath}
\usepackage{ifxetex,ifluatex}
\usepackage{fixltx2e} % provides \textsubscript
\ifnum 0\ifxetex 1\fi\ifluatex 1\fi=0 % if pdftex
  \usepackage[T1]{fontenc}
  \usepackage[utf8]{inputenc}
  \usepackage{textcomp} % provides euro and other symbols
\else % if luatex or xelatex
  \usepackage{unicode-math}
  \defaultfontfeatures{Ligatures=TeX,Scale=MatchLowercase}
\fi
% use upquote if available, for straight quotes in verbatim environments
\IfFileExists{upquote.sty}{\usepackage{upquote}}{}
% use microtype if available
\IfFileExists{microtype.sty}{%
\usepackage[]{microtype}
\UseMicrotypeSet[protrusion]{basicmath} % disable protrusion for tt fonts
}{}
\IfFileExists{parskip.sty}{%
\usepackage{parskip}
}{% else
\setlength{\parindent}{0pt}
\setlength{\parskip}{6pt plus 2pt minus 1pt}
}
\usepackage{hyperref}
\hypersetup{
            pdftitle={R\_Markdown Tutorial test},
            pdfauthor={AK},
            pdfborder={0 0 0},
            breaklinks=true}
\urlstyle{same}  % don't use monospace font for urls
\usepackage[margin=1in]{geometry}
\usepackage{color}
\usepackage{fancyvrb}
\newcommand{\VerbBar}{|}
\newcommand{\VERB}{\Verb[commandchars=\\\{\}]}
\DefineVerbatimEnvironment{Highlighting}{Verbatim}{commandchars=\\\{\}}
% Add ',fontsize=\small' for more characters per line
\usepackage{framed}
\definecolor{shadecolor}{RGB}{248,248,248}
\newenvironment{Shaded}{\begin{snugshade}}{\end{snugshade}}
\newcommand{\AlertTok}[1]{\textcolor[rgb]{0.94,0.16,0.16}{#1}}
\newcommand{\AnnotationTok}[1]{\textcolor[rgb]{0.56,0.35,0.01}{\textbf{\textit{#1}}}}
\newcommand{\AttributeTok}[1]{\textcolor[rgb]{0.77,0.63,0.00}{#1}}
\newcommand{\BaseNTok}[1]{\textcolor[rgb]{0.00,0.00,0.81}{#1}}
\newcommand{\BuiltInTok}[1]{#1}
\newcommand{\CharTok}[1]{\textcolor[rgb]{0.31,0.60,0.02}{#1}}
\newcommand{\CommentTok}[1]{\textcolor[rgb]{0.56,0.35,0.01}{\textit{#1}}}
\newcommand{\CommentVarTok}[1]{\textcolor[rgb]{0.56,0.35,0.01}{\textbf{\textit{#1}}}}
\newcommand{\ConstantTok}[1]{\textcolor[rgb]{0.00,0.00,0.00}{#1}}
\newcommand{\ControlFlowTok}[1]{\textcolor[rgb]{0.13,0.29,0.53}{\textbf{#1}}}
\newcommand{\DataTypeTok}[1]{\textcolor[rgb]{0.13,0.29,0.53}{#1}}
\newcommand{\DecValTok}[1]{\textcolor[rgb]{0.00,0.00,0.81}{#1}}
\newcommand{\DocumentationTok}[1]{\textcolor[rgb]{0.56,0.35,0.01}{\textbf{\textit{#1}}}}
\newcommand{\ErrorTok}[1]{\textcolor[rgb]{0.64,0.00,0.00}{\textbf{#1}}}
\newcommand{\ExtensionTok}[1]{#1}
\newcommand{\FloatTok}[1]{\textcolor[rgb]{0.00,0.00,0.81}{#1}}
\newcommand{\FunctionTok}[1]{\textcolor[rgb]{0.00,0.00,0.00}{#1}}
\newcommand{\ImportTok}[1]{#1}
\newcommand{\InformationTok}[1]{\textcolor[rgb]{0.56,0.35,0.01}{\textbf{\textit{#1}}}}
\newcommand{\KeywordTok}[1]{\textcolor[rgb]{0.13,0.29,0.53}{\textbf{#1}}}
\newcommand{\NormalTok}[1]{#1}
\newcommand{\OperatorTok}[1]{\textcolor[rgb]{0.81,0.36,0.00}{\textbf{#1}}}
\newcommand{\OtherTok}[1]{\textcolor[rgb]{0.56,0.35,0.01}{#1}}
\newcommand{\PreprocessorTok}[1]{\textcolor[rgb]{0.56,0.35,0.01}{\textit{#1}}}
\newcommand{\RegionMarkerTok}[1]{#1}
\newcommand{\SpecialCharTok}[1]{\textcolor[rgb]{0.00,0.00,0.00}{#1}}
\newcommand{\SpecialStringTok}[1]{\textcolor[rgb]{0.31,0.60,0.02}{#1}}
\newcommand{\StringTok}[1]{\textcolor[rgb]{0.31,0.60,0.02}{#1}}
\newcommand{\VariableTok}[1]{\textcolor[rgb]{0.00,0.00,0.00}{#1}}
\newcommand{\VerbatimStringTok}[1]{\textcolor[rgb]{0.31,0.60,0.02}{#1}}
\newcommand{\WarningTok}[1]{\textcolor[rgb]{0.56,0.35,0.01}{\textbf{\textit{#1}}}}
\usepackage{longtable,booktabs}
% Fix footnotes in tables (requires footnote package)
\IfFileExists{footnote.sty}{\usepackage{footnote}\makesavenoteenv{longtable}}{}
\usepackage{graphicx,grffile}
\makeatletter
\def\maxwidth{\ifdim\Gin@nat@width>\linewidth\linewidth\else\Gin@nat@width\fi}
\def\maxheight{\ifdim\Gin@nat@height>\textheight\textheight\else\Gin@nat@height\fi}
\makeatother
% Scale images if necessary, so that they will not overflow the page
% margins by default, and it is still possible to overwrite the defaults
% using explicit options in \includegraphics[width, height, ...]{}
\setkeys{Gin}{width=\maxwidth,height=\maxheight,keepaspectratio}
\setlength{\emergencystretch}{3em}  % prevent overfull lines
\providecommand{\tightlist}{%
  \setlength{\itemsep}{0pt}\setlength{\parskip}{0pt}}
\setcounter{secnumdepth}{0}
% Redefines (sub)paragraphs to behave more like sections
\ifx\paragraph\undefined\else
\let\oldparagraph\paragraph
\renewcommand{\paragraph}[1]{\oldparagraph{#1}\mbox{}}
\fi
\ifx\subparagraph\undefined\else
\let\oldsubparagraph\subparagraph
\renewcommand{\subparagraph}[1]{\oldsubparagraph{#1}\mbox{}}
\fi

% set default figure placement to htbp
\makeatletter
\def\fps@figure{htbp}
\makeatother


\title{R\_Markdown Tutorial test}
\author{AK}
\date{2/2/2020}

\begin{document}
\maketitle

\begin{Shaded}
\begin{Highlighting}[]
\NormalTok{norm <-}\StringTok{ }\KeywordTok{rnorm}\NormalTok{(}\DecValTok{100}\NormalTok{, }\DataTypeTok{mean =} \DecValTok{0}\NormalTok{, }\DataTypeTok{sd =} \DecValTok{1}\NormalTok{)}
\end{Highlighting}
\end{Shaded}

\includegraphics{R_Markdown_Tutorial_Test_files/figure-latex/unnamed-chunk-2-1.pdf}

\begin{Shaded}
\begin{Highlighting}[]
\NormalTok{A <-}\StringTok{ }\KeywordTok{c}\NormalTok{(}\StringTok{"a"}\NormalTok{, }\StringTok{"a"}\NormalTok{, }\StringTok{"b"}\NormalTok{, }\StringTok{"b"}\NormalTok{)}
\NormalTok{B <-}\StringTok{ }\KeywordTok{c}\NormalTok{(}\DecValTok{5}\NormalTok{, }\DecValTok{10}\NormalTok{, }\DecValTok{15}\NormalTok{, }\DecValTok{20}\NormalTok{)}
\NormalTok{dataframe <-}\StringTok{ }\KeywordTok{data.frame}\NormalTok{(A, B)}
\KeywordTok{plot}\NormalTok{(dataframe)}
\end{Highlighting}
\end{Shaded}

\includegraphics{R_Markdown_Tutorial_Test_files/figure-latex/unnamed-chunk-3-1.pdf}

\begin{Shaded}
\begin{Highlighting}[]
\NormalTok{dataframe}
\end{Highlighting}
\end{Shaded}

\begin{verbatim}
##   A  B
## 1 a  5
## 2 a 10
## 3 b 15
## 4 b 20
\end{verbatim}

\begin{Shaded}
\begin{Highlighting}[]
\KeywordTok{library}\NormalTok{(knitr)}
\KeywordTok{kable}\NormalTok{(dataframe, }\DataTypeTok{digits =} \DecValTok{2}\NormalTok{)}
\end{Highlighting}
\end{Shaded}

\begin{longtable}[]{@{}lr@{}}
\toprule
A & B\tabularnewline
\midrule
\endhead
a & 5\tabularnewline
a & 10\tabularnewline
b & 15\tabularnewline
b & 20\tabularnewline
\bottomrule
\end{longtable}

\begin{Shaded}
\begin{Highlighting}[]
\KeywordTok{library}\NormalTok{(pander)}
\NormalTok{plant <-}\StringTok{ }\KeywordTok{c}\NormalTok{(}\StringTok{"a"}\NormalTok{, }\StringTok{"b"}\NormalTok{, }\StringTok{"c"}\NormalTok{)}
\NormalTok{temperature <-}\StringTok{ }\KeywordTok{c}\NormalTok{(}\DecValTok{20}\NormalTok{, }\DecValTok{20}\NormalTok{, }\DecValTok{20}\NormalTok{)}
\NormalTok{growth <-}\StringTok{ }\KeywordTok{c}\NormalTok{(}\FloatTok{0.65}\NormalTok{, }\FloatTok{0.95}\NormalTok{, }\FloatTok{0.15}\NormalTok{)}
\NormalTok{dataframe <-}\StringTok{ }\KeywordTok{data.frame}\NormalTok{(plant, temperature, growth)}
\KeywordTok{emphasize.italics.cols}\NormalTok{(}\DecValTok{3}\NormalTok{)   }\CommentTok{# Make the 3rd column italics}
\KeywordTok{pander}\NormalTok{(dataframe)           }\CommentTok{# Create the table}
\end{Highlighting}
\end{Shaded}

\begin{longtable}[]{@{}ccc@{}}
\toprule
\begin{minipage}[b]{0.10\columnwidth}\centering
plant\strut
\end{minipage} & \begin{minipage}[b]{0.18\columnwidth}\centering
temperature\strut
\end{minipage} & \begin{minipage}[b]{0.11\columnwidth}\centering
growth\strut
\end{minipage}\tabularnewline
\midrule
\endhead
\begin{minipage}[t]{0.10\columnwidth}\centering
a\strut
\end{minipage} & \begin{minipage}[t]{0.18\columnwidth}\centering
20\strut
\end{minipage} & \begin{minipage}[t]{0.11\columnwidth}\centering
\emph{0.65}\strut
\end{minipage}\tabularnewline
\begin{minipage}[t]{0.10\columnwidth}\centering
b\strut
\end{minipage} & \begin{minipage}[t]{0.18\columnwidth}\centering
20\strut
\end{minipage} & \begin{minipage}[t]{0.11\columnwidth}\centering
\emph{0.95}\strut
\end{minipage}\tabularnewline
\begin{minipage}[t]{0.10\columnwidth}\centering
c\strut
\end{minipage} & \begin{minipage}[t]{0.18\columnwidth}\centering
20\strut
\end{minipage} & \begin{minipage}[t]{0.11\columnwidth}\centering
\emph{0.15}\strut
\end{minipage}\tabularnewline
\bottomrule
\end{longtable}

\begin{Shaded}
\begin{Highlighting}[]
\KeywordTok{library}\NormalTok{(broom)}
\NormalTok{A <-}\StringTok{ }\KeywordTok{c}\NormalTok{(}\DecValTok{20}\NormalTok{, }\DecValTok{15}\NormalTok{, }\DecValTok{10}\NormalTok{)}
\NormalTok{B <-}\StringTok{ }\KeywordTok{c}\NormalTok{(}\DecValTok{1}\NormalTok{, }\DecValTok{2}\NormalTok{, }\DecValTok{3}\NormalTok{)}

\NormalTok{lm_test <-}\StringTok{ }\KeywordTok{lm}\NormalTok{(A }\OperatorTok{~}\StringTok{ }\NormalTok{B)            }\CommentTok{# Creating linear model }
\KeywordTok{summary}\NormalTok{(lm_test)                }\CommentTok{# Obtaining linear model summary statistics}
\end{Highlighting}
\end{Shaded}

\begin{verbatim}
## Warning in summary.lm(lm_test): essentially perfect fit: summary may be
## unreliable
\end{verbatim}

\begin{verbatim}
## 
## Call:
## lm(formula = A ~ B)
## 
## Residuals:
##          1          2          3 
##  1.088e-15 -2.176e-15  1.088e-15 
## 
## Coefficients:
##               Estimate Std. Error    t value Pr(>|t|)    
## (Intercept)  2.500e+01  4.070e-15  6.142e+15  < 2e-16 ***
## B           -5.000e+00  1.884e-15 -2.654e+15  2.4e-16 ***
## ---
## Signif. codes:  0 '***' 0.001 '**' 0.01 '*' 0.05 '.' 0.1 ' ' 1
## 
## Residual standard error: 2.665e-15 on 1 degrees of freedom
## Multiple R-squared:      1,  Adjusted R-squared:      1 
## F-statistic: 7.043e+30 on 1 and 1 DF,  p-value: 2.399e-16
\end{verbatim}

\begin{Shaded}
\begin{Highlighting}[]
\NormalTok{table_obj <-}\StringTok{ }\KeywordTok{tidy}\NormalTok{(lm_test)      }\CommentTok{# Using tidy() to create a new R object called table }
\end{Highlighting}
\end{Shaded}

\begin{verbatim}
## Warning in summary.lm(x): essentially perfect fit: summary may be unreliable
\end{verbatim}

\begin{Shaded}
\begin{Highlighting}[]
\KeywordTok{pander}\NormalTok{(table_obj, }\DataTypeTok{digits =} \DecValTok{3}\NormalTok{)   }\CommentTok{# Using pander() to view the created table, with 3 sig figs }
\end{Highlighting}
\end{Shaded}

\begin{longtable}[]{@{}ccccc@{}}
\toprule
\begin{minipage}[b]{0.17\columnwidth}\centering
term\strut
\end{minipage} & \begin{minipage}[b]{0.13\columnwidth}\centering
estimate\strut
\end{minipage} & \begin{minipage}[b]{0.14\columnwidth}\centering
std.error\strut
\end{minipage} & \begin{minipage}[b]{0.14\columnwidth}\centering
statistic\strut
\end{minipage} & \begin{minipage}[b]{0.14\columnwidth}\centering
p.value\strut
\end{minipage}\tabularnewline
\midrule
\endhead
\begin{minipage}[t]{0.17\columnwidth}\centering
(Intercept)\strut
\end{minipage} & \begin{minipage}[t]{0.13\columnwidth}\centering
25\strut
\end{minipage} & \begin{minipage}[t]{0.14\columnwidth}\centering
4.07e-15\strut
\end{minipage} & \begin{minipage}[t]{0.14\columnwidth}\centering
6.14e+15\strut
\end{minipage} & \begin{minipage}[t]{0.14\columnwidth}\centering
1.04e-16\strut
\end{minipage}\tabularnewline
\begin{minipage}[t]{0.17\columnwidth}\centering
B\strut
\end{minipage} & \begin{minipage}[t]{0.13\columnwidth}\centering
-5\strut
\end{minipage} & \begin{minipage}[t]{0.14\columnwidth}\centering
1.88e-15\strut
\end{minipage} & \begin{minipage}[t]{0.14\columnwidth}\centering
-2.65e+15\strut
\end{minipage} & \begin{minipage}[t]{0.14\columnwidth}\centering
2.4e-16\strut
\end{minipage}\tabularnewline
\bottomrule
\end{longtable}

\emph{Italics} \textbf{Bold} \#Header 1 \#\#Header 2 1. Ordered list
item

\end{document}
